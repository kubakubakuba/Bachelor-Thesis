%!TEX root = ../main.tex

\appendix
\renewcommand\chaptername{Appendix}

\renewcommand{\thechapter}{A}
\renewcommand\chaptername{Appendix A}

\chapter{Attachments}
\noindent
Below is a list of attachments to this thesis.

\vspace{1em}

\renewcommand{\arraystretch}{2.2} % optional for row height

\noindent
\begin{tabularx}{\textwidth}{>{\raggedright\arraybackslash}p{7cm} X}
\texttt{mrs-uvdar-distance-estimator.zip} & code used in the event-based camera response section of this thesis \\
\texttt{metavision-pyocamcalib.zip} &  code used for the calibration of lenses, and the generation of calibration frames \\
\texttt{ros-event-distance.zip} &  ROS implementation of a pose estimator \\
\end{tabularx}

\vspace{2em}

\noindent All source code related to this thesis is also publicly available in the repositories listed below.

\vspace{1em}

\renewcommand{\arraystretch}{2.2} % optional for row height

\noindent
\begin{tabularx}{\textwidth}{>{\raggedright\arraybackslash}p{5cm} X}
\hline
\textbf{Thesis \LaTeX\ Source} & \href{https://github.com/kubakubakuba/Bachelor-Thesis}{github.com/kubakubakuba/Bachelor-Thesis} \\
\textbf{Response Analysis} & \href{https://github.com/kubakubakuba/mrs-uvdar-distance-estimator}{github.com/kubakubakuba/mrs-uvdar-distance-estimator} \\
\textbf{Calibration Scripts} & \href{https://github.com/kubakubakuba/metavision-pyocamcalib}{github.com/kubakubakuba/metavision-pyocamcalib} \\
\textbf{ROS DistanceEstimator} & \href{https://github.com/kubakubakuba/ros-event-distance}{github.com/kubakubakuba/ros-event-distance} \\
\hline
\end{tabularx}

\renewcommand{\thechapter}{B}
\renewcommand\chaptername{Appendix B}

\chapter{Used AI Software}

\noindent In accordance with the \texttt{Methodological guideline No. 5/2023}
\footnote{\url{https://www.cvut.cz/sites/default/files/content/d1dc93cd-5894-4521-b799-c7e715d3c59e/en/20231003-methodological-guideline-no-52023.pdf}}
the following software was used during the writing of this thesis:
\begin{itemize}
    \item {Writefull\footnote{\url{https://www.writefull.com/}} for rewording in the online Overleaf\footnote{\url{https://www.overleaf.com/}} editor}
    \item {GitHub Copilot\footnote{\url{https://github.com/features/copilot}} for code completion}
    \item {ChatGPT\footnote{\url{https://chat.openai.com/}} for wording suggestions and feedback, for plotting help}
\end{itemize}