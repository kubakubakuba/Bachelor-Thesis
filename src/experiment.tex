%!TEX root = ../main.tex

\chapter{Experiment\label{chap:experiment}}

The experiment was performed with two X500 \ac{UAV}s, each of them equipped with a Prophesee EVK4 event-based camera, one with a 2.5mm f/1.6
fish eye lens with an \ac{FOV} of 93.5 degrees and the second one with Entaniya 1.07mm f/2.8 fish eye lens with an \ac{FOV} of 280 degrees.
Each UAV is also equipped with a basler camera with a fish eye lens, to provide normal video signal that is recorded alongside the event stream
from the event-based camera.
Both cameras are connected to the onboard Intel NUC computer running the ROS system, on which all the processing is done during the flight. Both 
\ac{UAV}s are also equipped with a \ac{RTK} module, which is used to localize the \ac{UAV}, and is used as ground truth data for the pose estimation.
The measurements were collected during the \ac{MRS} Camp in Temešvár in August 2025, the \ac{UAV}s can be seen on \reffig{fig:uav33_37}.

\begin{figure}[H]
	\centering
	\subfloat[UAV33] {
	  \includegraphics[width=0.4\textwidth]{./fig/photos/uav33.jpg}
	  \label{fig:uav33}
	}
	\subfloat[UAV37] {
	  \includegraphics[width=0.4\textwidth]{./fig/photos/uav37.jpg}
	  \label{fig:uav37}
	}
	\caption{
		Two X500 UAVs, UAV33 on \reffig{fig:uav33} and UAV37 on \reffig{fig:uav37}.
  }
	\label{fig:uav33_37}
\end{figure}
Two pilots were manually flying with the \ac{UAV}s, periodically changing the distance and incidence angle between the ....

TODO: SHOW MEASURED DATA FROM RQT

TODO: SHOW GNSS/ESTIMATION DIFFERENCES

TODO: SHOW THE RVIZ/RQTPLOT VISUALIZATION PIPELINE