%!TEX root = ../main.tex

\chapter{Conclusion\label{chap:conclusion}}
In this thesis, the response of an event-based camera to a modulated source of light (a \ac{UV} \ac{LED}) was discussed.
It was shown how the changing frequency, distance, and incidence angle influenced the average number of events generated by the camera.
The calibration method for fisheye lenses by Scaramuzza et al.~\cite{scaramuzzacalibration} was then discussed, which, in our case,
utilized an LED lattice target instead of the normally used chessboard image pattern.

Distance estimation was performed using the \ac{P3P} algorithm, which estimated the rotation and translation of a known
arrangement of 3D marker locations using their 2D image coordinates. A stationary data set was collected and analyzed with this
approach, which has shown a median distance estimation error of $0.34$ meters with a standard deviation of $0.16$ meters.
A \texttt{DistanceEstimator} node was implemented in \ac{ROS}, which eased the distance estimation by subscribing
to detected \ac{LED} locations and publishing the estimated pose and distance.

This approach was used in our real-life experiment, where data was collected from two flying \ac{UAV}s, which were manually controlled by two pilots.
The data was then analyzed afterward, by manually marking the \ac{LED} locations. The resulting estimate has shown
higher errors (a mean absolute error of 2.47 meters with a standard deviation of 1.75 meters), compared to the statically measured data set,
due to the increased relative distance during flight and physical limitations of the methods and equipment used.

\section{Future work}
In the future, the detection of the \ac{LED}s is planned to be automated, enabling real-time distance and pose estimation.
This could be achieved by incorporating a robust feature detection algorithm or a machine learning-based object detection
algorithm into the pipeline.
A method using \ac{RSSR} for the improvement of the pose estimation could also be used to further improve pose estimation precision,
which was not further explored in this thesis, as for the inconclusive results obtained during the trial analysis of the data measured for this method and the lack of time for further research.