%!TEX root = ../main.tex

\chapter{Conclusion\label{chap:conclusion}}

In this thesis, we discussed the response of an event-based camera to a modulated source of light - a \ac{UV} \ac{LED}.
We have shown how the changing frequency, distance, and incidence angle influence the average number of events generated
by the camera. We then discussed the calibration method for fish eye lenses by Scaramuzza et al. \cite{scaramuzzacalibration},
which in our cases uses an \ac{LED} lattice target, instead of a normally used checkerboard.

The distance estimation was done using the \ac{P3P} algorithm, that estimated the rotation and translation of a known
arrangement of 3D locations using their image coordinates. We collected a stationary data set and analyzed it with this
approach. We implemented a \texttt{DistanceEstimator} node in \ac{ROS}, that eases the distance estimation by subscribing
to detected \ac{LED} locations and publishing the estimated pose and distance.

We used this approach in our real-life experiment, where we collected data from two flying \ac{UAV}s, manually controlled by two pilots. We then analyzed the after measuring them, manually marking the \ac{LED} locations. The resulting estimate showed
higher errors, when compared to the statically measured data set, due to increased distance during flying and physical
limitations of the methods and equipment used.

\section{Future work}
In the future, we plan to automate the detection of the \ac{LED}s, enabling real-time distance and pose estimation.
This could be achieved by incorporating a robust feature detection algorithm or a machine learning-based object detection
algorithm into the pipeline.
\todo{maybe write more}