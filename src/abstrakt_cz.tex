%!TEX root = ../main.tex

\begin{changemargin}{0.8cm}{0.8cm}

~\vfill{}

\section*{Abstrakt}
\vskip 0.5em

\sloppy

Tato práce představuje způsob odhadu relativní polohy objektu vybaveného modulovanými světelnými emitory za použití
eventové kamery jako našeho senzoru. Provedený přístup začíná detekcí a shlukováním asynchronních událostí do "blobů", pro které
jsou vypočteny jejich centroidy. Spojením těchto centroidů s již známou geometrií světelných emitorů je vypočtena
rotace a translace objektu pomocí algoritmu Perspective-n-Point.
Pro zajištění přesných měření je nejprve potřeba provést kalibraci kamery a objektivu, která zahrnuje snímání
kalibrační mřížky s rovnoměrně rozmístěnými diodami, aby byly zajištěny jasné referenční body během kalibrace.
Experimentální ověření na statických a pohybujících se kvadrokoptérách ukazuje průměrnou lokalizační přesnost v řádu desítek centimetrů ve stacionárních scénářích a přesnost na úrovni jednotek metrů na velkých vzdálenostech během aktivního pohybu.

\vskip 1em

{\bf Klíčová slova} \KlicovaSlova

\vskip 2.5cm

\end{changemargin}